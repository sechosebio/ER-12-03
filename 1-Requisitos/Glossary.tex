%%%%%%%%%%%%%%%%%%%%%%%%%%%%%%%%%%%%%%%%%%%%%%%%%%%%%%%%%%%%%%%%%%%%%%%%%%%%%%%%
% FUENTE
%%%%%%%%%%%%%%%%%%%%%%%%%%%%%%%%%%%%%%%%%%%%%%%%%%%%%%%%%%%%%%%%%%%%%%%%%%%%%%%%

% Plantilla creada por Eduardo Mosqueira Rey a partir de un original de 
% Rational Software Corporation

%%%%%%%%%%%%%%%%%%%%%%%%%%%%%%%%%%%%%%%%%%%%%%%%%%%%%%%%%%%%%%%%%%%%%%%%%%%%%%%%
% CONFIGURACIÓN TEXSTUDIO DEL CORRECTOR ORTOGRÁFICO
%%%%%%%%%%%%%%%%%%%%%%%%%%%%%%%%%%%%%%%%%%%%%%%%%%%%%%%%%%%%%%%%%%%%%%%%%%%%%%%%

% !TeX spellcheck = en_US
% Usar el lenguaje es_ES para la corrección en castellano

%%%%%%%%%%%%%%%%%%%%%%%%%%%%%%%%%%%%%%%%%%%%%%%%%%%%%%%%%%%%%%%%%%%%%%%%%%%%%%%%
% TIPO DE DOCUMENTO Y PAQUETES
%%%%%%%%%%%%%%%%%%%%%%%%%%%%%%%%%%%%%%%%%%%%%%%%%%%%%%%%%%%%%%%%%%%%%%%%%%%%%%%%

\documentclass[12pt, a4paper, titlepage]{article}

% Indicamos que el lenguaje es el español 
\usepackage[spanish]{babel} % Soporte multilenguaje para LaTeX.
\usepackage[a4paper, top=2.5cm, bottom=2.5cm, left=2.5cm, right=2.5cm]{geometry} % Interfaz flexible para definir las dimensiones del documento
\usepackage[utf8]{inputenc} % Aceptar diferentes tipos de codificación de caracteres de entrada (en este caso usamos la codificación Unicode UTF-8)
\usepackage{graphicx} % Soporte aumentado para gráficos 
\usepackage{color} % Para usar colores
\usepackage{hyperref} % Para manejar referencias cruzadas. P.ej. añadir hiperenlaces al índice

\begin{document}

%%%%%%%%%%%%%%%%%%%%%%%%%%%%%%%%%%%%%%%%%%%%%%%%%%%%%%%%%%%%%%%%%%%%%%%%%%%%%%%%
% PORTADA
%%%%%%%%%%%%%%%%%%%%%%%%%%%%%%%%%%%%%%%%%%%%%%%%%%%%%%%%%%%%%%%%%%%%%%%%%%%%%%%%

\begin{titlepage}

\includegraphics[width=15cm]{Imagenes/Simbolo_logo_UDC.png}

% Lista de tamaños: \Huge, \huge, \LARGE, \Large, \large, \small, \footnotesize, \tiny
\vspace{3cm}

\begin{center}
\includegraphics[scale=0.3]{Imagenes/1a_Practica_ER_14-15.png}
\end{center}


\begin{flushright}
	
	\LARGE{\textbf{ JoinMe!}}
	
	\LARGE{\textbf{Glosario}}
	
	\large{\textbf{Version 1.1}}
\end{flushright}
\vspace{1cm}
\begin{center}
José Antonio López Sebio\\
Pablo Paz Varela\\
Grupo ER-12-03\\
\end{center}


\vspace{2cm}
\begin{center}
	\large{\textbf{Revision History}}
	
    \begin{tabular}{ | p{4cm} | p{2cm} | p{5cm} | p{4cm} |}
    \hline
    \textbf{Date} & \textbf{Version} & \textbf{Description} & \textbf{Author} \\ \hline
    05/03/2015 & 1.0 & Versión inicial & ER-12-03  \\ \hline
    06/03/2015&  1.1 & Versión final & ER-12-03 \\ \hline
    & & & \\ \hline
    \end{tabular}
\end{center}

\end{titlepage}
\clearpage

%%%%%%%%%%%%%%%%%%%%%%%%%%%%%%%%%%%%%%%%%%%%%%%%%%%%%%%%%%%%%%%%%%%%%%%%%%%%%%%%
% INDICE
%%%%%%%%%%%%%%%%%%%%%%%%%%%%%%%%%%%%%%%%%%%%%%%%%%%%%%%%%%%%%%%%%%%%%%%%%%%%%%%%

\tableofcontents
\clearpage

%%%%%%%%%%%%%%%%%%%%%%%%%%%%%%%%%%%%%%%%%%%%%%%%%%%%%%%%%%%%%%%%%%%%%%%%%%%%%%%%
\section{Introducción}
%%%%%%%%%%%%%%%%%%%%%%%%%%%%%%%%%%%%%%%%%%%%%%%%%%%%%%%%%%%%%%%%%%%%%%%%%%%%%%%%

%**********************************************************
\subsection{Objetivo}
%**********************************************************

El objetivo de este documento es aclarar al lector el significado de algunos términos utilizados en los documentos del proyecto. Podemos considerarlo un diccionario informal que define una terminología consistente para todo el personal interesado.

%**********************************************************
\subsection{Alcance}
%**********************************************************

Este glosario se aplica a la red social \textbf{JoinMe!}. Se definen aquellos términos que puedan ser poco claros, ambiguos o requieran elaboración.

%**********************************************************
\subsection{Referencias}
%**********************************************************

Este documento hace referencia a:\\

1. Visión de JoinMe!, v1.2, 2015, ER-12-03

2. Modelo de Casos de Uso de JoinMe!, v1.4, 2015, ER-12-03

3. Especificación Suplementaria de JoinMe!, v1.1, 2015, ER-12-03

%**********************************************************
\subsection{Visión general}
%**********************************************************

A continuación se presentan las definiciones de los términos en orden alfabético.

%%%%%%%%%%%%%%%%%%%%%%%%%%%%%%%%%%%%%%%%%%%%%%%%%%%%%%%%%%%%%%%%%%%%%%%%%%%%%%%%
\section{Definiciones}
%%%%%%%%%%%%%%%%%%%%%%%%%%%%%%%%%%%%%%%%%%%%%%%%%%%%%%%%%%%%%%%%%%%%%%%%%%%%%%%%

%**********************************************************
\subsection{Administrador}
%**********************************************************

Los administradores son personas que crean y gestionan la actividad de grupos.

%**********************************************************
\subsection{Amigo}
%**********************************************************

Los amigos son personas con las que te conectas y con las que compartes tu actividad en \textbf{JoinMe!}.

%**********************************************************
\subsection{Círculo}
%**********************************************************

Los círculos definen un grupo de amigos. Nos permiten clasificar a nuestros amigos, para así poder controlar fácilmente con quienes compartimos nuestra actividad.

%**********************************************************
\subsection{Entrada}
%**********************************************************

Las entradas son publicaciones en nuestro muro, y pueden ser de varios tipos: un texto, un enlace a una web, un lugar, una fotografía o un vídeo.

%**********************************************************
\subsection{Etiqueta}
%**********************************************************

Una etiqueta vincula a una persona con una fotografía. Por ejemplo, puedes etiquetar a alguien para indicar que sale en la imagen.

%**********************************************************
\subsection{Grupo}
%**********************************************************

Los grupos son espacios privados en los que reunir a un conjunto de gente con algo en común.

%**********************************************************
\subsection{JoinMe! Coins}
%**********************************************************

Es la moneda virtual de \textbf{JoinMe!}, y con ella se pueden comprar distintos productos ofertados por la red social o realizar transferencias entre amigos. Se obtienen con dinero real.

%**********************************************************
\subsection{Página de inicio}
%**********************************************************

Es una página de scroll infinito en la que se refleja la actividad relacionada con tus amigos, eliminando la necesidad de consultar sus muros individualmente.

%**********************************************************
\subsection{Perfil}
%**********************************************************

Es la pagina donde se muestran la información que compartes, tu frase de estado, tu muro, tus fotografías, videos, etc.

%**********************************************************
\subsection{Perfil verificado}
%**********************************************************

Los perfiles verificados son aquellos cuya identidad puede ser asegurada a través de un certificado digital. El resto de usuarios verán un icono en la cuenta indicando que pueden confiar en esa identidad.

%**********************************************************
\subsection{TDD}
%**********************************************************

Test Driven Development, o Desarrollo Guiado por Pruebas, es una práctica de programación que consiste en codificar las pruebas antes que la propia implementación de la funcionalidad que van a probar.

%**********************************************************
\subsection{W3C}
%**********************************************************

El World Wide Web Consortium es una comunidad internacional que desarrolla estándares que aseguran el crecimiento de la Web a largo plazo.


\end{document}