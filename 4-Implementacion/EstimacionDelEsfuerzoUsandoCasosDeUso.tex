% Plantilla creada por Eduardo Mosqueira Rey
%
% Libro online bastante completo para consulta de Latex: http://en.wikibooks.org/wiki/LaTeX/
% Versión en castellano: http://es.wikibooks.org/wiki/Manual_de_LaTeX
\documentclass[12pt, a4paper, titlepage]{article}

\usepackage[spanish]{babel} % Soporte multilenguaje para LaTeX.

\usepackage[a4paper, top=2.5cm, bottom=2.5cm, left=2.5cm, right=2.5cm]{geometry} % Interfaz flexible para definir las dimensiones del documento

\usepackage[utf8]{inputenc} % Aceptar diferentes tipos de codificación de caracteres de entrada (en este caso usamos la codificación Unicode UTF-8)

\usepackage{graphicx} % Soporte aumentado para gráficos 
\usepackage{hyperref}
\usepackage{longtable}
\begin{document}

%%%%%%%%%%%%%%%%%%%%%%%%%%%%%%%%%%%%%%%%%%%%%%%%%%%%%%%%%%%%%%%%%%%%%%%%%%%%%%%%
% PORTADA
%%%%%%%%%%%%%%%%%%%%%%%%%%%%%%%%%%%%%%%%%%%%%%%%%%%%%%%%%%%%%%%%%%%%%%%%%%%%%%%%

\begin{titlepage}

\includegraphics[width=15cm]{Imagenes/Simbolo_logo_UDC.png}

% Lista de tamaños: \Huge, \huge, \LARGE, \Large, \large, \small, \footnotesize, \tiny
\vspace{3cm}

\begin{center}
\includegraphics[scale=0.3]{Imagenes/1a_Practica_ER_14-15.png}
\end{center}


\begin{flushright}
	
	\LARGE{\textbf{ JoinMe!}}
	
	\LARGE{\textbf{Estimación del esfuerzo usando Puntos de Casos de Uso
	}}
	
	\large{\textbf{Version 1.0}}
	
\end{flushright}

\vspace{1cm}
\begin{center}
José Antonio López Sebio\\
Pablo Paz Varela\\
Grupo ER-12-03\\
\end{center}

\vspace{2cm}

\begin{center}
	\large{\textbf{Histórico}}
	
    \begin{tabular}{ | p{4cm} | p{2cm} | p{6cm} | p{3cm} |}
    \hline
    \textbf{Fecha} & \textbf{Version} & \textbf{Descripción} & \textbf{Autor} \\ \hline
      06/05/2014 & 1.0 & Primera Revisión & ER-12-03\\ \hline
      &  &  & ER-12-03\\ \hline
     &  & &\\ \hline
    \end{tabular}
\end{center}


\end{titlepage}
\clearpage

%%%%%%%%%%%%%%%%%%%%%%%%%%%%%%%%%%%%%%%%%%%%%%%%%%%%%%%%%%%%%%%%%%%%%%%%%%%%%%%%
% INDICE
%%%%%%%%%%%%%%%%%%%%%%%%%%%%%%%%%%%%%%%%%%%%%%%%%%%%%%%%%%%%%%%%%%%%%%%%%%%%%%%%

\tableofcontents
\clearpage

%%%%%%%%%%%%%%%%%%%%%%%%%%%%%%%%%%%%%%%%%%%%%%%%%%%%%%%%%%%%%%%%%%%%%%%%%%%%%%%%
\section{Introducción}
%%%%%%%%%%%%%%%%%%%%%%%%%%%%%%%%%%%%%%%%%%%%%%%%%%%%%%%%%%%%%%%%%%%%%%%%%%%%%%%%

%**********************************************************
\subsection{Propósito}
%**********************************************************


%**********************************************************
\subsection{Alcance}
%**********************************************************

%**********************************************************
\subsection{Definiciones, Acrónimos y Abreviaturas}
%**********************************************************

Ver Glosario

%**********************************************************
\subsection{Referencias}
%**********************************************************

\begin{itemize}

    \item JoinMe! - Glosario
    \item JoinMe! - Modelo de casos de uso
    \item JoinMe! - Especificación suplementaria
    \item JoinMe! - Modelo del dominio
\end{itemize}

\section{UUCW: Unadjusted Use Case Weight}

\begin{table}[h]
	\begin{center}
		\begin{tabular}{|c|c|c|}
	\hline
	Caso de Uso & Número de Transacciones (o clases) & Peso \\ \hline
	Simple & De 1 a 3 transacciones (menos de 5 clases) & 5 \\ \hline
	Medio & De 4 a 7 transacciones (de 5 a 10 clases) & 10 \\ \hline
	Complejo & 8 o más transacciones (más de 10 clases) & 15 \\ \hline
	
	\end{tabular}
	\caption{Clasificación Casos de Uso}
	\end{center}
\end{table}


	\begin{center}
		\begin{longtable}{|c|c|}
			\hline
			Caso de Uso & Categoría \\ \hline
			\endfirsthead
			\hline
			Caso de Uso & Categoría \\ 
			\endhead
			CU01 - Registrar usuario & Simple \\ \hline
			CU02 - Modificación de perfil & Simple \\ \hline
			CU03 - Aceptar solicitud de amistad & Simple \\ \hline
			CU04 - Rechazar solicitud de amistad & Simple \\ \hline
			CU05 - Ver amigos & Simple \\ \hline
			CU06 - Enviar solicitud de amistad & Simple \\ \hline
			CU07 - Crear círculo & Simple \\ \hline
			CU08 - Borrar círculo & Simple \\ \hline
			CU09 - Añadir amigo a círculo & Simple \\ \hline
			CU10 - Eliminar amigo de un círculo & Simple \\ \hline
			CU11 - Cambiar frase de estado & Simple \\ \hline
			CU12 - Publicar acción en el muro & Simple \\ \hline
			CU13 - Crear entrada & Complejo \\ \hline
			CU14 - Borrar entrada & Simple \\ \hline
			CU15 - Editar entrada & Complejo \\ \hline
			CU16 - Crear grupo & Simple \\ \hline
			CU17 - Eliminar grupo & Simple \\ \hline
			CU18 - Ingresar en un grupo & Simple \\ \hline
			CU19 - Salir de un grupo & Simple \\ \hline
			CU20 - Publicar noticia en un grupo & Simple \\ \hline
			CU21 - Crear hilo en un foro de discusión de grupo & Simple \\ \hline
			CU22 - Responder en un hilo del grupo & Simple \\ \hline
			CU23 - Crear álbum de fotografías & Simple \\ \hline
			CU24 - Subir fotografía & Simple \\ \hline
			CU25 - Borrar álbum de fotografías & Simple \\ \hline
			CU26 - Borrar fotografía & Simple \\ \hline
			CU27 - Etiquetar contacto en fotografía & Simple \\ \hline
			CU28 - Borrar etiqueta de fotografía & Simple \\ \hline
			CU29 - Edición sencilla de fotos & Medio \\ \hline
			CU30 - Subir vídeo & Simple \\ \hline
			CU31 - Borrar vídeo & Simple \\ \hline
			CU32 - Comentar una publicación & Simple \\ \hline
			CU33 - Borrar comentario & Simple \\ \hline
			CU34 - Enviar mensaje privado & Simple \\ \hline
			CU35 - Mostrar página de início & Simple \\ \hline
			CU36 - Filtrar información en página de inicio & Medio \\ \hline
			CU37 - Subir ficheros & Simple \\ \hline
			CU38 - Borrar ficheros & Simple \\ \hline
			CU39 - Comprar \textit{JoinMe! Coins} & Simple \\ \hline
			CU40 - Regalar \textit{JoinMe! Coins} & Simple \\ \hline
			CU41 - Consultar catálogo de productos & Simple \\ \hline
			CU42 - Comprar un producto & Simple \\ \hline
			CU43 - Indicar relevancia de un anuncio & Simple \\ \hline
			CU44 - Crear un anuncio & Simple \\ \hline
		
		\caption{Clasificación Casos de Uso} 
		\end{longtable}
		
	\end{center}



\section{UAW: Unadjusted Actor Weight}

\begin{table}[h]
	\begin{center}
		\begin{tabular}{|c|p{12cm}|c|}
	\hline
	Actor & Descripción  & Peso \\ \hline
	Simple & Sistema externo que interactúa con el sistema a través de un API bien definido & 1 \\ \hline
	Medio & Sistema externo que interactúa con el sistema a través de protocolos de comunicación (TCP/IP, FTP, HTTP, base de datos, etc)  & 2 \\ \hline
	Complejo & Actor humano usando un interfaz gráfico de la aplicación & 3 \\ \hline
	
	\end{tabular}
	\caption{Clasificación Actores}
	\end{center}
\end{table}

\begin{table}[h]
	\begin{center}
		\begin{tabular}{|c|c|}
			\hline
			Actor & Categoría \\ \hline
			Usuario & Complejo \\ \hline
			Empresa & Complejo \\ \hline
			
		\end{tabular}
		\caption{Clasificación Actores}
	\end{center}
\end{table}

\section{TCF: Technical Complexity Factor}

\begin{table}[h]
	\begin{center}
		\begin{tabular}{|c|p{10cm}|c|c|c|}
			\hline
			Factor & Descripción & Peso & Valor & Impacto \\ \hline
			T1 & Sistema distribuido & 2 & 5 & 10 \\ \hline
			T2 & Objetivos de performance o tiempo de respuesta & 1 & 5 & 5 \\ \hline
			T3 & Eficiencia del usuario final & 1 & 3 & 3 \\ \hline
			T4 & Procesamiento interno complejo & 1 & 3 & 3 \\ \hline
			T5 & El código debe ser reutilizable & 1 & 3 & 3 \\ \hline
			T6 & Facilidad de instalción & 0.5 & 2 & 1 \\ \hline
			T7 & Facilidad de uso & 0.5 & 5 & 2.5 \\ \hline
			T8 & Portabilidad & 2 & 4 & 8 \\ \hline
			T9 & Facilidad de cambio & 1 & 4 & 4 \\ \hline
			T10 & Concurrencia & 1 & 3 & 3 \\ \hline
			T11 & Incluye objetivos especiales de seguridad & 1 & 5 & 5 \\ \hline
			T12 & Provee acceso directo a terceras partes & 1 & 1 & 1 \\ \hline
			T13 & Se requiere facilidades especiales de entrenamiento a usuario & 1 & 1 & 1 \\ \hline
			& & & TF = & 49.5 \\ \hline

	
	
	\end{tabular}
	\caption{Technical Factor}
	\end{center}
\end{table}

TCF = 0.6 + (49.5 / 100) = 1.095

\section{ECF: Environmental Complexity Factor}

\begin{table}[h]
	\begin{center}
		\begin{tabular}{|c|p{10cm}|c|c|c|}
			\hline
			Factor & Descripción & Peso & Valor & Impacto \\ \hline
			E1 & Familiaridad con la metodología de desarrollo & 1.5 & 2 & 3 \\ \hline
			E2 & Experiencia en la aplicación & 0.5 & 0 & 0 \\ \hline
			E3 & Experiencia en orientación a objetos & 1 & 5 & 5 \\ \hline
			E4 & Capacidad del analista líder & 0.5 & 5 & 2.5 \\ \hline
			E5 & Motivación & 1 & 3 & 3 \\ \hline
			E6 & Estabilidad de los requisitos & 2 & 2 & 4 \\ \hline
			E7 & Personal a tiempo parcial  & -1 & 3 & -3 \\ \hline
			E8 & Dificultad del lenguaje de programación & -1 & 1 & -1 \\ \hline
			& & & EF = & 13.5 \\ \hline
			
			
			
		\end{tabular}
		\caption{Environmental Factor}
	\end{center}
\end{table}

ECF = 1.4 + (-0.03 * 13.5) = 0.995

\section{UCP: Use Case Point}

UCP = (UUCW + UAW) * TCF * ECF 
= (250 + 6) * 1.095 * 0.995 
= 278.9184

\section{Personas - Hora}

\begin{table}[h]
	\begin{center}
		\begin{tabular}{|c|c|c|}
			\hline
			UUCW & Unadjusted Use Case Weight & 250 \\ \hline
			UAW & Unadjusted Actor Weight & 6 \\ \hline
			TCF & Technical Complexity Factor & 1.095 \\ \hline
			ECF & Environmental Complexity Factor & 0.995 \\ \hline
			UCP & Use Case Point & 278.9184 \\ \hline
			PF &  Productivity Factor & 22 \\ \hline
			MH &  Man-Hour, Personas-Hora & 6136,2048 \\ \hline
			
		\end{tabular}
		\caption{Cálculos finales}
	\end{center}
\end{table}

\end{document}